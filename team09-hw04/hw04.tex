% Options for packages loaded elsewhere
\PassOptionsToPackage{unicode}{hyperref}
\PassOptionsToPackage{hyphens}{url}
%
\documentclass[
]{article}
\usepackage{amsmath,amssymb}
\usepackage{lmodern}
\usepackage{iftex}
\ifPDFTeX
  \usepackage[T1]{fontenc}
  \usepackage[utf8]{inputenc}
  \usepackage{textcomp} % provide euro and other symbols
\else % if luatex or xetex
  \usepackage{unicode-math}
  \defaultfontfeatures{Scale=MatchLowercase}
  \defaultfontfeatures[\rmfamily]{Ligatures=TeX,Scale=1}
\fi
% Use upquote if available, for straight quotes in verbatim environments
\IfFileExists{upquote.sty}{\usepackage{upquote}}{}
\IfFileExists{microtype.sty}{% use microtype if available
  \usepackage[]{microtype}
  \UseMicrotypeSet[protrusion]{basicmath} % disable protrusion for tt fonts
}{}
\makeatletter
\@ifundefined{KOMAClassName}{% if non-KOMA class
  \IfFileExists{parskip.sty}{%
    \usepackage{parskip}
  }{% else
    \setlength{\parindent}{0pt}
    \setlength{\parskip}{6pt plus 2pt minus 1pt}}
}{% if KOMA class
  \KOMAoptions{parskip=half}}
\makeatother
\usepackage{xcolor}
\IfFileExists{xurl.sty}{\usepackage{xurl}}{} % add URL line breaks if available
\IfFileExists{bookmark.sty}{\usepackage{bookmark}}{\usepackage{hyperref}}
\hypersetup{
  pdftitle={Quantitative Methods in Political Science - Homework 4},
  pdfauthor={Team member 1 (with percentage); Team member 2 (with percentage); Team member 3 (with percentage)},
  hidelinks,
  pdfcreator={LaTeX via pandoc}}
\urlstyle{same} % disable monospaced font for URLs
\usepackage[margin=1in]{geometry}
\usepackage{color}
\usepackage{fancyvrb}
\newcommand{\VerbBar}{|}
\newcommand{\VERB}{\Verb[commandchars=\\\{\}]}
\DefineVerbatimEnvironment{Highlighting}{Verbatim}{commandchars=\\\{\}}
% Add ',fontsize=\small' for more characters per line
\usepackage{framed}
\definecolor{shadecolor}{RGB}{248,248,248}
\newenvironment{Shaded}{\begin{snugshade}}{\end{snugshade}}
\newcommand{\AlertTok}[1]{\textcolor[rgb]{0.94,0.16,0.16}{#1}}
\newcommand{\AnnotationTok}[1]{\textcolor[rgb]{0.56,0.35,0.01}{\textbf{\textit{#1}}}}
\newcommand{\AttributeTok}[1]{\textcolor[rgb]{0.77,0.63,0.00}{#1}}
\newcommand{\BaseNTok}[1]{\textcolor[rgb]{0.00,0.00,0.81}{#1}}
\newcommand{\BuiltInTok}[1]{#1}
\newcommand{\CharTok}[1]{\textcolor[rgb]{0.31,0.60,0.02}{#1}}
\newcommand{\CommentTok}[1]{\textcolor[rgb]{0.56,0.35,0.01}{\textit{#1}}}
\newcommand{\CommentVarTok}[1]{\textcolor[rgb]{0.56,0.35,0.01}{\textbf{\textit{#1}}}}
\newcommand{\ConstantTok}[1]{\textcolor[rgb]{0.00,0.00,0.00}{#1}}
\newcommand{\ControlFlowTok}[1]{\textcolor[rgb]{0.13,0.29,0.53}{\textbf{#1}}}
\newcommand{\DataTypeTok}[1]{\textcolor[rgb]{0.13,0.29,0.53}{#1}}
\newcommand{\DecValTok}[1]{\textcolor[rgb]{0.00,0.00,0.81}{#1}}
\newcommand{\DocumentationTok}[1]{\textcolor[rgb]{0.56,0.35,0.01}{\textbf{\textit{#1}}}}
\newcommand{\ErrorTok}[1]{\textcolor[rgb]{0.64,0.00,0.00}{\textbf{#1}}}
\newcommand{\ExtensionTok}[1]{#1}
\newcommand{\FloatTok}[1]{\textcolor[rgb]{0.00,0.00,0.81}{#1}}
\newcommand{\FunctionTok}[1]{\textcolor[rgb]{0.00,0.00,0.00}{#1}}
\newcommand{\ImportTok}[1]{#1}
\newcommand{\InformationTok}[1]{\textcolor[rgb]{0.56,0.35,0.01}{\textbf{\textit{#1}}}}
\newcommand{\KeywordTok}[1]{\textcolor[rgb]{0.13,0.29,0.53}{\textbf{#1}}}
\newcommand{\NormalTok}[1]{#1}
\newcommand{\OperatorTok}[1]{\textcolor[rgb]{0.81,0.36,0.00}{\textbf{#1}}}
\newcommand{\OtherTok}[1]{\textcolor[rgb]{0.56,0.35,0.01}{#1}}
\newcommand{\PreprocessorTok}[1]{\textcolor[rgb]{0.56,0.35,0.01}{\textit{#1}}}
\newcommand{\RegionMarkerTok}[1]{#1}
\newcommand{\SpecialCharTok}[1]{\textcolor[rgb]{0.00,0.00,0.00}{#1}}
\newcommand{\SpecialStringTok}[1]{\textcolor[rgb]{0.31,0.60,0.02}{#1}}
\newcommand{\StringTok}[1]{\textcolor[rgb]{0.31,0.60,0.02}{#1}}
\newcommand{\VariableTok}[1]{\textcolor[rgb]{0.00,0.00,0.00}{#1}}
\newcommand{\VerbatimStringTok}[1]{\textcolor[rgb]{0.31,0.60,0.02}{#1}}
\newcommand{\WarningTok}[1]{\textcolor[rgb]{0.56,0.35,0.01}{\textbf{\textit{#1}}}}
\usepackage{longtable,booktabs,array}
\usepackage{calc} % for calculating minipage widths
% Correct order of tables after \paragraph or \subparagraph
\usepackage{etoolbox}
\makeatletter
\patchcmd\longtable{\par}{\if@noskipsec\mbox{}\fi\par}{}{}
\makeatother
% Allow footnotes in longtable head/foot
\IfFileExists{footnotehyper.sty}{\usepackage{footnotehyper}}{\usepackage{footnote}}
\makesavenoteenv{longtable}
\usepackage{graphicx}
\makeatletter
\def\maxwidth{\ifdim\Gin@nat@width>\linewidth\linewidth\else\Gin@nat@width\fi}
\def\maxheight{\ifdim\Gin@nat@height>\textheight\textheight\else\Gin@nat@height\fi}
\makeatother
% Scale images if necessary, so that they will not overflow the page
% margins by default, and it is still possible to overwrite the defaults
% using explicit options in \includegraphics[width, height, ...]{}
\setkeys{Gin}{width=\maxwidth,height=\maxheight,keepaspectratio}
% Set default figure placement to htbp
\makeatletter
\def\fps@figure{htbp}
\makeatother
\setlength{\emergencystretch}{3em} % prevent overfull lines
\providecommand{\tightlist}{%
  \setlength{\itemsep}{0pt}\setlength{\parskip}{0pt}}
\setcounter{secnumdepth}{-\maxdimen} % remove section numbering
\ifLuaTeX
  \usepackage{selnolig}  % disable illegal ligatures
\fi

\title{Quantitative Methods in Political Science - Homework 4}
\author{Team member 1 (with percentage) \and Team member 2 (with
percentage) \and Team member 3 (with percentage)}
\date{Due: October 12, 2021}

\begin{document}
\maketitle

{
\setcounter{tocdepth}{2}
\tableofcontents
}
\textbf{Notes:}

\begin{quote}
\textbf{Note:} If you do not have any special reason, please do not load
additional packages to solve this homework assignment. If you
nevertheless do so, please indicate why you think this is necessary and
add the package to the \texttt{p\_needed} vector in the setup chunk
above.
\end{quote}

\begin{quote}
In addition to the \texttt{Rmd}, please make sure that in the repo,
there is a PDF knitted from your code. The automated check for
reproducibility on Github will run only when you include the word
``final'' into the commit message.
\end{quote}

\begin{quote}
\emph{Please try to answer the questions with \textbf{short} but very
\textbf{precise} statements. However, do not hide behind seemingly fancy
jargon.}
\end{quote}

\hypertarget{part-1-linear-regression-basics}{%
\subsection{Part 1: Linear Regression
Basics}\label{part-1-linear-regression-basics}}

\textbf{1.1 For the following data set, calculate the OLS estimator of
slope and intercept \emph{by hand} (not in R, by hand on paper). Show
your calculations.}

\emph{Hint: Take a look at slide 14 from the lecture. You can write
LaTeX-code for mathematical formulas like this: \(0.5 = \frac{1}{2}\).}

\begin{longtable}[]{@{}cc@{}}
\caption{Simple Data Set}\tabularnewline
\toprule
x & y \\
\midrule
\endfirsthead
\toprule
x & y \\
\midrule
\endhead
3.2 & 1.8 \\
6 & 8.5 \\
4.5 & 4.8 \\
10 & 11.5 \\
12 & 14.5 \\
\bottomrule
\end{longtable}

\textbf{1.2 Then read in the data in R by hand. Write a function that
calculates the slope and intercept in \texttt{R} and apply it to the
data set.}

\emph{Hint: Create two vectors using the following command:
\texttt{x\ \textless{}-\ c(1,2,3,...)}.}

\textbf{1.3 Use the built-in regression functions in \texttt{R} and
compare the results to your calculations from 1.1 and the results in
1.2.}

\hypertarget{part-2-corruption-and-wealth}{%
\subsection{Part 2: Corruption and
Wealth}\label{part-2-corruption-and-wealth}}

\emph{Political and economic corruption is an annoying problem
throughout the world. How does the extent of corruption change with the
wealth of countries, if at all?}

\emph{Transparency International provides estimates of corruption levels
in countries of the world based on surveys of business people, risk
analysts and the general public. The index ranges between 10 (highly
clean) and 0 (highly corrupt). Your task is to assess the substantive
impact of wealth on corruption and evaluate the substantive relevance of
this effect.}

\emph{You will need the \texttt{corruption.dta} which is in the
\texttt{raw-data} folder of this repository.}

\textbf{2.1 The cross-sectional data set contains the average corruption
index between 2000 and 2007 and the (per capita) wealth variables in
2002. How many countries does the data set include? Which countries are
the least and most corrupt? Which countries are the least and most
wealthy?}

Answer:

\textbf{2.2 Run a regression of corruption on GDP per capita, with the
corruption score as the dependent variable. Make a nicely formatted (!)
table from the output. Knit the \texttt{.Rmd} file to a \texttt{.pdf}
file to check whether your table looks good. Write down the regression
line equation. What is the substantial meaning of the estimated
intercept and slope?}

Answer:

\textbf{2.3 Interpret the \emph{substantive relevance} of the results.
To do so, compare the predicted level of corruption in countries that
are at the 25th and 75th percentile of the GDP per capita range. Would
you describe the effect as large or small?}

Answer:

\textbf{2.4 Make a scatterplot of the corruption variable versus GDP per
capita. Make an informed decision regarding the axes selection. Label
the points with country names or abbreviations and add the regression
line.}

Answer:

\textbf{2.5 Imagine a fellow student seeks your advice. They generated
the labeled scatterplot from 2.4 and are unsure whether they should keep
the labels or not. What is your advice and why?}

Answer:

\textbf{2.6 What countries are unusually corrupt and lacking in
corruption given their level of GDP per capita? Study the residual
values of corruption (i.e., the values that cannot be explained or
predicted by using information about GDP per capita). Include a residual
plot, labeling any potentially interesting outliers.}

Answer:

\textbf{2.7 Interpret the goodness-of-fit of the model.}

Answer:

\textbf{2.8 Adjusted R-squared by hand. Translate the following formula
into R and compare your result to the output in the summary. How do you
explain the difference between the \(R^2\) and Adjusted \(R^2\) for your
model?}

\[Adj.R^2=1 - (1 - R^2)\frac{n-1}{n-k-1}\]

\hypertarget{r-code}{%
\section{R code}\label{r-code}}

\begin{Shaded}
\begin{Highlighting}[]
\CommentTok{\# The first line sets an option for the final document that can be produced from}
\CommentTok{\# the .Rmd file. Don\textquotesingle{}t worry about it.}
\NormalTok{knitr}\SpecialCharTok{::}\NormalTok{opts\_chunk}\SpecialCharTok{$}\FunctionTok{set}\NormalTok{(}\AttributeTok{echo =} \ConstantTok{TRUE}\NormalTok{)}

\CommentTok{\# The next bit (lines 22{-}43) is quite powerful and useful. }
\CommentTok{\# First you define which packages you need for your analysis and assign it to }
\CommentTok{\# the p\_needed object. }
\NormalTok{p\_needed }\OtherTok{\textless{}{-}}
  \FunctionTok{c}\NormalTok{(}\StringTok{"viridis"}\NormalTok{, }\StringTok{"stargazer"}\NormalTok{)}

\CommentTok{\# Now you check which packages are already installed on your computer.}
\CommentTok{\# The function installed.packages() returns a vector with all the installed }
\CommentTok{\# packages.}
\NormalTok{packages }\OtherTok{\textless{}{-}} \FunctionTok{rownames}\NormalTok{(}\FunctionTok{installed.packages}\NormalTok{())}
\CommentTok{\# Then you check which of the packages you need are not installed on your }
\CommentTok{\# computer yet. Essentially you compare the vector p\_needed with the vector}
\CommentTok{\# packages. The result of this comparison is assigned to p\_to\_install.}
\NormalTok{p\_to\_install }\OtherTok{\textless{}{-}}\NormalTok{ p\_needed[}\SpecialCharTok{!}\NormalTok{(p\_needed }\SpecialCharTok{\%in\%}\NormalTok{ packages)]}
\CommentTok{\# If at least one element is in p\_to\_install you then install those missing}
\CommentTok{\# packages.}
\ControlFlowTok{if}\NormalTok{ (}\FunctionTok{length}\NormalTok{(p\_to\_install) }\SpecialCharTok{\textgreater{}} \DecValTok{0}\NormalTok{) \{}
  \FunctionTok{install.packages}\NormalTok{(p\_to\_install)}
\NormalTok{\}}
\CommentTok{\# Now that all packages are installed on the computer, you can load them for}
\CommentTok{\# this project. Additionally the expression returns whether the packages were}
\CommentTok{\# successfully loaded.}
\FunctionTok{sapply}\NormalTok{(p\_needed, require, }\AttributeTok{character.only =} \ConstantTok{TRUE}\NormalTok{)}

\CommentTok{\# This is an option for stargazer tables}
\CommentTok{\# It automatically adapts the output to html or latex,}
\CommentTok{\# depending on whether we want a html or pdf file}
\NormalTok{stargazer\_opt }\OtherTok{\textless{}{-}} \FunctionTok{ifelse}\NormalTok{(knitr}\SpecialCharTok{::}\FunctionTok{is\_latex\_output}\NormalTok{(), }\StringTok{"latex"}\NormalTok{, }\StringTok{"html"}\NormalTok{)}
\end{Highlighting}
\end{Shaded}

\hypertarget{r-code-1}{%
\section{R code}\label{r-code-1}}

\begin{Shaded}
\begin{Highlighting}[]
\CommentTok{\# The first line sets an option for the final document that can be produced from}
\CommentTok{\# the .Rmd file. Don\textquotesingle{}t worry about it.}
\NormalTok{knitr}\SpecialCharTok{::}\NormalTok{opts\_chunk}\SpecialCharTok{$}\FunctionTok{set}\NormalTok{(}\AttributeTok{echo =} \ConstantTok{TRUE}\NormalTok{)}

\CommentTok{\# The next bit (lines 22{-}43) is quite powerful and useful. }
\CommentTok{\# First you define which packages you need for your analysis and assign it to }
\CommentTok{\# the p\_needed object. }
\NormalTok{p\_needed }\OtherTok{\textless{}{-}}
  \FunctionTok{c}\NormalTok{(}\StringTok{"viridis"}\NormalTok{, }\StringTok{"stargazer"}\NormalTok{)}

\CommentTok{\# Now you check which packages are already installed on your computer.}
\CommentTok{\# The function installed.packages() returns a vector with all the installed }
\CommentTok{\# packages.}
\NormalTok{packages }\OtherTok{\textless{}{-}} \FunctionTok{rownames}\NormalTok{(}\FunctionTok{installed.packages}\NormalTok{())}
\CommentTok{\# Then you check which of the packages you need are not installed on your }
\CommentTok{\# computer yet. Essentially you compare the vector p\_needed with the vector}
\CommentTok{\# packages. The result of this comparison is assigned to p\_to\_install.}
\NormalTok{p\_to\_install }\OtherTok{\textless{}{-}}\NormalTok{ p\_needed[}\SpecialCharTok{!}\NormalTok{(p\_needed }\SpecialCharTok{\%in\%}\NormalTok{ packages)]}
\CommentTok{\# If at least one element is in p\_to\_install you then install those missing}
\CommentTok{\# packages.}
\ControlFlowTok{if}\NormalTok{ (}\FunctionTok{length}\NormalTok{(p\_to\_install) }\SpecialCharTok{\textgreater{}} \DecValTok{0}\NormalTok{) \{}
  \FunctionTok{install.packages}\NormalTok{(p\_to\_install)}
\NormalTok{\}}
\CommentTok{\# Now that all packages are installed on the computer, you can load them for}
\CommentTok{\# this project. Additionally the expression returns whether the packages were}
\CommentTok{\# successfully loaded.}
\FunctionTok{sapply}\NormalTok{(p\_needed, require, }\AttributeTok{character.only =} \ConstantTok{TRUE}\NormalTok{)}

\CommentTok{\# This is an option for stargazer tables}
\CommentTok{\# It automatically adapts the output to html or latex,}
\CommentTok{\# depending on whether we want a html or pdf file}
\NormalTok{stargazer\_opt }\OtherTok{\textless{}{-}} \FunctionTok{ifelse}\NormalTok{(knitr}\SpecialCharTok{::}\FunctionTok{is\_latex\_output}\NormalTok{(), }\StringTok{"latex"}\NormalTok{, }\StringTok{"html"}\NormalTok{)}
\end{Highlighting}
\end{Shaded}


\end{document}
