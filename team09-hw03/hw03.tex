% Options for packages loaded elsewhere
\PassOptionsToPackage{unicode}{hyperref}
\PassOptionsToPackage{hyphens}{url}
%
\documentclass[
]{article}
\usepackage{amsmath,amssymb}
\usepackage{lmodern}
\usepackage{iftex}
\ifPDFTeX
  \usepackage[T1]{fontenc}
  \usepackage[utf8]{inputenc}
  \usepackage{textcomp} % provide euro and other symbols
\else % if luatex or xetex
  \usepackage{unicode-math}
  \defaultfontfeatures{Scale=MatchLowercase}
  \defaultfontfeatures[\rmfamily]{Ligatures=TeX,Scale=1}
\fi
% Use upquote if available, for straight quotes in verbatim environments
\IfFileExists{upquote.sty}{\usepackage{upquote}}{}
\IfFileExists{microtype.sty}{% use microtype if available
  \usepackage[]{microtype}
  \UseMicrotypeSet[protrusion]{basicmath} % disable protrusion for tt fonts
}{}
\makeatletter
\@ifundefined{KOMAClassName}{% if non-KOMA class
  \IfFileExists{parskip.sty}{%
    \usepackage{parskip}
  }{% else
    \setlength{\parindent}{0pt}
    \setlength{\parskip}{6pt plus 2pt minus 1pt}}
}{% if KOMA class
  \KOMAoptions{parskip=half}}
\makeatother
\usepackage{xcolor}
\IfFileExists{xurl.sty}{\usepackage{xurl}}{} % add URL line breaks if available
\IfFileExists{bookmark.sty}{\usepackage{bookmark}}{\usepackage{hyperref}}
\hypersetup{
  pdftitle={Quantitative Methods in Political Science - Homework 3},
  pdfauthor={Team member 1 (with percentage); Team member 2 (with percentage); Team member 3 (with percentage)},
  hidelinks,
  pdfcreator={LaTeX via pandoc}}
\urlstyle{same} % disable monospaced font for URLs
\usepackage[margin=1in]{geometry}
\usepackage{color}
\usepackage{fancyvrb}
\newcommand{\VerbBar}{|}
\newcommand{\VERB}{\Verb[commandchars=\\\{\}]}
\DefineVerbatimEnvironment{Highlighting}{Verbatim}{commandchars=\\\{\}}
% Add ',fontsize=\small' for more characters per line
\usepackage{framed}
\definecolor{shadecolor}{RGB}{248,248,248}
\newenvironment{Shaded}{\begin{snugshade}}{\end{snugshade}}
\newcommand{\AlertTok}[1]{\textcolor[rgb]{0.94,0.16,0.16}{#1}}
\newcommand{\AnnotationTok}[1]{\textcolor[rgb]{0.56,0.35,0.01}{\textbf{\textit{#1}}}}
\newcommand{\AttributeTok}[1]{\textcolor[rgb]{0.77,0.63,0.00}{#1}}
\newcommand{\BaseNTok}[1]{\textcolor[rgb]{0.00,0.00,0.81}{#1}}
\newcommand{\BuiltInTok}[1]{#1}
\newcommand{\CharTok}[1]{\textcolor[rgb]{0.31,0.60,0.02}{#1}}
\newcommand{\CommentTok}[1]{\textcolor[rgb]{0.56,0.35,0.01}{\textit{#1}}}
\newcommand{\CommentVarTok}[1]{\textcolor[rgb]{0.56,0.35,0.01}{\textbf{\textit{#1}}}}
\newcommand{\ConstantTok}[1]{\textcolor[rgb]{0.00,0.00,0.00}{#1}}
\newcommand{\ControlFlowTok}[1]{\textcolor[rgb]{0.13,0.29,0.53}{\textbf{#1}}}
\newcommand{\DataTypeTok}[1]{\textcolor[rgb]{0.13,0.29,0.53}{#1}}
\newcommand{\DecValTok}[1]{\textcolor[rgb]{0.00,0.00,0.81}{#1}}
\newcommand{\DocumentationTok}[1]{\textcolor[rgb]{0.56,0.35,0.01}{\textbf{\textit{#1}}}}
\newcommand{\ErrorTok}[1]{\textcolor[rgb]{0.64,0.00,0.00}{\textbf{#1}}}
\newcommand{\ExtensionTok}[1]{#1}
\newcommand{\FloatTok}[1]{\textcolor[rgb]{0.00,0.00,0.81}{#1}}
\newcommand{\FunctionTok}[1]{\textcolor[rgb]{0.00,0.00,0.00}{#1}}
\newcommand{\ImportTok}[1]{#1}
\newcommand{\InformationTok}[1]{\textcolor[rgb]{0.56,0.35,0.01}{\textbf{\textit{#1}}}}
\newcommand{\KeywordTok}[1]{\textcolor[rgb]{0.13,0.29,0.53}{\textbf{#1}}}
\newcommand{\NormalTok}[1]{#1}
\newcommand{\OperatorTok}[1]{\textcolor[rgb]{0.81,0.36,0.00}{\textbf{#1}}}
\newcommand{\OtherTok}[1]{\textcolor[rgb]{0.56,0.35,0.01}{#1}}
\newcommand{\PreprocessorTok}[1]{\textcolor[rgb]{0.56,0.35,0.01}{\textit{#1}}}
\newcommand{\RegionMarkerTok}[1]{#1}
\newcommand{\SpecialCharTok}[1]{\textcolor[rgb]{0.00,0.00,0.00}{#1}}
\newcommand{\SpecialStringTok}[1]{\textcolor[rgb]{0.31,0.60,0.02}{#1}}
\newcommand{\StringTok}[1]{\textcolor[rgb]{0.31,0.60,0.02}{#1}}
\newcommand{\VariableTok}[1]{\textcolor[rgb]{0.00,0.00,0.00}{#1}}
\newcommand{\VerbatimStringTok}[1]{\textcolor[rgb]{0.31,0.60,0.02}{#1}}
\newcommand{\WarningTok}[1]{\textcolor[rgb]{0.56,0.35,0.01}{\textbf{\textit{#1}}}}
\usepackage{graphicx}
\makeatletter
\def\maxwidth{\ifdim\Gin@nat@width>\linewidth\linewidth\else\Gin@nat@width\fi}
\def\maxheight{\ifdim\Gin@nat@height>\textheight\textheight\else\Gin@nat@height\fi}
\makeatother
% Scale images if necessary, so that they will not overflow the page
% margins by default, and it is still possible to overwrite the defaults
% using explicit options in \includegraphics[width, height, ...]{}
\setkeys{Gin}{width=\maxwidth,height=\maxheight,keepaspectratio}
% Set default figure placement to htbp
\makeatletter
\def\fps@figure{htbp}
\makeatother
\setlength{\emergencystretch}{3em} % prevent overfull lines
\providecommand{\tightlist}{%
  \setlength{\itemsep}{0pt}\setlength{\parskip}{0pt}}
\setcounter{secnumdepth}{-\maxdimen} % remove section numbering
\ifLuaTeX
  \usepackage{selnolig}  % disable illegal ligatures
\fi

\title{Quantitative Methods in Political Science - Homework 3}
\author{Team member 1 (with percentage) \and Team member 2 (with
percentage) \and Team member 3 (with percentage)}
\date{Due: October 5, 2021}

\begin{document}
\maketitle

\begin{quote}
\textbf{Note:} If you do not have any special reason, please do not load
additional packages to solve this homework assignment. If you
nevertheless do so, please indicate why you think this is necessary and
add the package to the \texttt{p\_needed} vector in the setup chunk
above.
\end{quote}

\hypertarget{part-1-definitions}{%
\subsection{Part 1: Definitions}\label{part-1-definitions}}

\textbf{1.1 What is a sampling distribution? If you consider a sampling
distribution of the sample mean, what can you say about its shape,
center, and spread?}

Answer:

\textbf{1.2 What is a confidence interval? Which assumptions does it
rely on?}

Answer:

\hypertarget{part-2-confidence-intervals}{%
\subsection{Part 2: Confidence
Intervals}\label{part-2-confidence-intervals}}

\emph{You collected a random sample of 50 students and asked them to
rate chancellor Merkel on a scale from 1 to 5, with 5 being the highest
rating. The mean score is \(2.3\) and the sample has a standard
deviation of \(0.86\).}

\textbf{2.1 Estimate Merkel's rating for the population of students and
calculate a 95\% and a 99\% confidence interval analytically. Interpret
these confidence intervals.}

Answer:

\textbf{2.2 Imagine you ran the survey again and had obtained the same
information from a random sample of 150 students instead of 50.
Calculate the 95\% confidence interval analytically and interpret this
interval. How and why does it differ from the confidence interval in
2.1?}

Answer:

\textbf{2.3 Consider the confidence interval in 2.2: If you ran the
survey again, can you expect with 95\% confidence that the average
rating of Merkel in this new sample will lie within the 95\% confidence
interval from 2.2?}

Answer:

\textbf{2.4 Repeat 2.1 and 2.2, but this time construct the confidence
interval using simulation. Plot the resulting distribution. Is it
different from the analytical one? If so, how and why?}

Answer:

\hypertarget{part-3-smart-mannheim-students}{%
\subsection{Part 3: Smart Mannheim
Students}\label{part-3-smart-mannheim-students}}

\emph{In the general population IQ is distributed normally with a mean
of 100 and a standard deviation of 19. You take a simple random sample
of 40 students in Mannheim and find that their mean IQ is 117.}

\textbf{3.1 Calculate the standard error of the mean. What does this
value tell you?}

Answer:

\textbf{3.2 Someone claims that on average, Mannheim students have a
higher IQ than the general population. Given your data, do you agree?}

Answer:

\textbf{3.3 What is more likely: observing a sample with mean IQ of 117
or observing an individual with an IQ of 117?}

Answer:

\textbf{3.4 Now suppose you got the IQ scores of 40 students who receive
scholarships for academic excellence instead of a random sample. Does
reporting the confidence intervals for the mean IQ address the problem
of bias in sampling?}

Answer:

\textbf{3.5 Suppose that, based on a different sample, 95\% confidence
interval for the mean IQ of Mannheim students was calculated as (109,
125). How would you evaluate the following interpretations of this
confidence interval? Say whether you think the interpretation is correct
or not and explain why you think so.}

\begin{quote}
\emph{95\% of the time the mean IQ of Mannheim students in this sample
is between 109 and 125, if we were to draw repeated samples.}
\end{quote}

Answer:

\begin{quote}
\emph{95\% of all students in Mannheim have IQ between 109 and 125.}
\end{quote}

Answer:

\begin{quote}
\emph{We are 95\% confident that the mean IQ of all students in Mannheim
is between 109 and 125.}
\end{quote}

Answer:

\begin{quote}
\emph{We are 95\% confident that the mean IQ of all students in this
sample is between 109 and 125.}
\end{quote}

Answer:

\begin{quote}
\emph{If we repeated the experiment a large number of times, there is a
95\% chance that the mean IQ of Mannheim students is between 109 and 125
in each time.}
\end{quote}

Answer:

\hypertarget{part-4-proportions}{%
\subsection{Part 4: Proportions}\label{part-4-proportions}}

\emph{Suppose that a military dictator in an unnamed country holds a
plebiscite (a yes/no vote of confidence) and claims that he was
supported by 68\% of the voters. A human rights group suspects foul play
and hires you to test the validity of the dictator's claim. You have a
budget that allows you to randomly sample 180 voters from the country.
You collect your sample of 180, and you find that 95 people actually
voted yes.}

\textbf{4.1 You decide to have a closer look at the sample and explore
if there are significant differences between men and women support rates
of the dictator. In your sample, 80 respondents are female and 45\% of
women reported voting \emph{yes}.}

\textbf{4.1.1 What is the share of men voting for the dictator?}

Answer:

\textbf{4.1.2 Calculate the difference between proportions of supporters
among men and women and 95\% confidence intervals for this difference
using simulation. What is your conclusion: do men, on average, tend to
support the dictator more than women?}

Answer:

\textbf{4.1.3 Calculate the standard error of the difference from
simulation. Compare this standard error for proportion difference with
analytical one:}

\(SE = \sqrt{\frac{p_1(1-p_1)}{n_1} + \frac{p_2(1-p_2)}{n_2}}\)

\textbf{4.2 (Optional) Given the information from the sample, what is
the probability that at least 68\% of the population voted yes?}

\textbf{4.3 (Optional) What is the probability that a majority of people
in the country support the dictator?}

\hypertarget{r-code}{%
\section{R code}\label{r-code}}

\begin{Shaded}
\begin{Highlighting}[]
\CommentTok{\# The first line sets an option for the final document that can be produced from}
\CommentTok{\# the .Rmd file. Don\textquotesingle{}t worry about it.}
\NormalTok{knitr}\SpecialCharTok{::}\NormalTok{opts\_chunk}\SpecialCharTok{$}\FunctionTok{set}\NormalTok{(}\AttributeTok{echo =} \ConstantTok{TRUE}\NormalTok{)}

\CommentTok{\# The next bit (lines 22{-}43) is quite powerful and useful. }
\CommentTok{\# First you define which packages you need for your analysis and assign it to }
\CommentTok{\# the p\_needed object. }
\NormalTok{p\_needed }\OtherTok{\textless{}{-}} \FunctionTok{c}\NormalTok{(}\StringTok{"viridis"}\NormalTok{) }\CommentTok{\# add your packages here }

\CommentTok{\# Now you check which packages are already installed on your computer.}
\CommentTok{\# The function installed.packages() returns a vector with all the installed }
\CommentTok{\# packages.}
\NormalTok{packages }\OtherTok{\textless{}{-}} \FunctionTok{rownames}\NormalTok{(}\FunctionTok{installed.packages}\NormalTok{())}

\CommentTok{\# Then you check which of the packages you need are not installed on your }
\CommentTok{\# computer yet. Essentially you compare the vector p\_needed with the vector}
\CommentTok{\# packages. The result of this comparison is assigned to p\_to\_install.}
\NormalTok{p\_to\_install }\OtherTok{\textless{}{-}}\NormalTok{ p\_needed[}\SpecialCharTok{!}\NormalTok{(p\_needed }\SpecialCharTok{\%in\%}\NormalTok{ packages)]}

\CommentTok{\# If at least one element is in p\_to\_install you then install those missing}
\CommentTok{\# packages.}
\ControlFlowTok{if}\NormalTok{ (}\FunctionTok{length}\NormalTok{(p\_to\_install) }\SpecialCharTok{\textgreater{}} \DecValTok{0}\NormalTok{) \{}
  \FunctionTok{install.packages}\NormalTok{(p\_to\_install)}
\NormalTok{\}}
\CommentTok{\# Now that all packages are installed on the computer, you can load them for}
\CommentTok{\# this project. Additionally the expression returns whether the packages were}
\CommentTok{\# successfully loaded.}
\FunctionTok{sapply}\NormalTok{(p\_needed, require, }\AttributeTok{character.only =} \ConstantTok{TRUE}\NormalTok{)}
\end{Highlighting}
\end{Shaded}


\end{document}
